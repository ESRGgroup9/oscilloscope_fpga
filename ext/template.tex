%!TEX TS-program = xelatex
\documentclass[11pt,a4paper]{report}
\usepackage[margin=1in]{geometry}
\geometry{a4paper, right=25mm, rmargin=0mm, outer=25mm, left=0mm, lmargin=0mm, inner=25mm,
	top=0mm, tmargin=25mm, bottom=0mm, bmargin=25mm,
}

%Packages
%\usepackage{sectsty}   %Overleaf, com LuaTEX não recomenda o uso (gera warnings)
\usepackage{graphicx}
\usepackage{blindtext}
\usepackage{fontspec}
\usepackage{multirow}
\usepackage{afterpage}
\usepackage{layout}
\usepackage{blindtext}
\usepackage{fancyhdr}
\usepackage{xcolor}
\usepackage{color}
\usepackage{apacite}
\usepackage{caption}
\usepackage{fix-cm}
\usepackage{float}
\usepackage{makecell}
\usepackage{tabularx}
\usepackage{titlesec}
\usepackage{etoolbox}
\usepackage[acronym]{glossaries}
\usepackage{pdfpages}
\usepackage{longtable}
%\usepackage{hyperref} %hifens
%*****hifens*******
\tolerance=1
\emergencystretch=\maxdimen
\hyphenpenalty=10000
\hbadness=10000
%*****hyperrefs*******
\usepackage[portuguese, english]{babel}


%Definições dos Títulos
\titleformat{\chapter}[display]
{\normalfont\bfseries\color{darkgray}}{}{0pt}{\Large}
\titleformat*{\section}{\color{darkgray}\Large\bfseries}
\titleformat*{\subsection}{\color{darkgray}\Large\bfseries\color{darkgray}}
\titleformat*{\subsubsection}{\color{darkgray}\Large\bfseries}
\setcounter{secnumdepth}{3}
\renewcommand{\thesubsubsection}{\alph{subsubsection})}

%Definições do Documento
\renewcommand{\baselinestretch}{1.5} 
\setmainfont{NewsGotT.ttf}
\setmainfont[
%Mapping=tex-text,
Ligatures=TeX,      % Indicação do Overleaf
AutoFakeSlant=0.20,
BoldFont=NewsGotTBold.ttf
]{NewsGotT.ttf} 

\titlespacing{\chapter}{0pt}{-40pt}{10pt} 

%Definições das Figuras
\renewcommand\thefigure{\arabic{figure}}
\setcounter{figure}{0}

%Definições das Tabelas
\renewcommand\thetable{\arabic{table}}
\renewcommand{\tablename}{}
\renewcommand{\thetable}{}
\setcounter{table}{0}
\floatstyle{plaintop}
\restylefloat{table}
\captionsetup[table]{labelsep=space}

%*****************************************************************
% %Definições do Cabeçalho
% \fancyhf{}
% \renewcommand{\chaptermark}[1]{\markboth{#1}{}}
%\renewcommand{\sectionmark}[1]{\markright{#1}{}}
%\fancyhead[R]{\textcolor{gray}{Análise Crítica de Modelos Simples para a Abordagem à Cibersegurança e Privacidade}} 

% %Definições das Secções
% \fancyhead[L]{}
% \setlength{\headheight}{15pt}
% \pagestyle{fancy}
% \patchcmd{\chapter}{\thispagestyle{plain}}{\thispagestyle{fancy}}{}{}
% \patchcmd{\headrule}{\hrule}{\color{black}\hrule}{}{} %Change color Hrule

%*******************************************************************

%Glossário
\input{chapter/glossario.tex}

%Acrónimos
\input{chapter/acronimos.tex}

\begin{document}
	\fancyfoot[C]{\thepage}
	\pagenumbering{roman}
	
	%Capa 
	\begin{titlepage}
%		\includepdf[pages={1},fitpaper]{chapter/pdf/capa_Dissertacao.pdf}
%		\includepdf[pages={1},fitpaper]{chapter/pdf/contracapa_Dissertacao.pdf}
%		\includepdf[pages={1}]{chapter/pdf/Folha_de_Rosto_Dissertacao.pdf}
	\end{titlepage}
	
	%\newpage\null\thispagestyle{empty}\newpage
	\setcounter{page}{2}
%	\includepdf[pages={1},pagecommand={}]{chapter/pdf/DIREITOS_AUTOR.pdf}
	
	%Agradecimentos
	\chapter*{Agradecimentos}
%	\input{chapter/agradecimentos.tex}
	
%	\includepdf[pages={1},pagecommand={}]{chapter/pdf/Declaracao_Integridade.pdf}
	
%	\input{chapter/__frase.tex}
	
	%Resumo
	\chapter*{Resumo}
%	\input{chapter/resumo.tex}
	%\includepdf[pages={1},pagecommand={}]{chapter/pdf/Resumo_Abstract.pdf}
	
	%\newpage\null\thispagestyle{empty}\newpage
	
	%Abstract
	\chapter*{Abstract}
%	\input{chapter/abstract.tex}
	%\includepdf[pages={2},pagecommand={}]{chapter/pdf/Resumo_Abstract.pdf}
	
	%\newpage\null\thispagestyle{empty}\newpage
	
	%Índice
	\renewcommand{\contentsname}{Índice}
	\tableofcontents
	\clearpage
	\let\oldnumberline\numberline% Copy \numberline into \oldnumberline
	\renewcommand{\numberline}[1]{\hspace*{-1.5em}}% Remove number argument
	
	%Índice de Figuras
	\renewcommand*\listfigurename{Índice de Figuras}
	\listoffigures
	\clearpage
	
	%Índice de Tabelas
	\renewcommand*{\listtablename}{Índice de Tabelas}
	\listoftables
	\clearpage
	
	%Siglas e Acrónimos
	\printglossary[type=\acronymtype,nonumberlist, title={Acrónimos}]
	%\clearpage
	
	%Glossary
	\printglossary[title={Glossário}, nonumberlist]
	
	%Definições do Rodapé
	%\fancyfoot[R]{\thepage}
	
	%Introdução
	\setcounter{chapter}{1}
%	\input{chapter/introducao.tex} %Added from chapter directory
	
	%\newpage\null\thispagestyle{empty}\newpage
	
	%Revisão de Literatura
	\setcounter{chapter}{2}
	\setcounter{section}{0}
%	\input{chapter/revisao_literatura.tex}
	
	%Abordagem Metedológica
	\setcounter{chapter}{3} %ALTERAR NA DISSERTAÇÂO
	\setcounter{section}{0}
%	\input{chapter/abordagem_metodologica.tex}
	
	%Trabalho Realizado
	\setcounter{chapter}{4}
	\setcounter{section}{0}
%	\input{chapter/trabalho_realizado.tex}
	
	%Planeamento
	\setcounter{chapter}{5} %ALTERAR NA DISSERTAÇÂO
	\setcounter{section}{0}
%	\input{chapter/plano_de_atividades.tex}
	
	%Conclusão
	\setcounter{chapter}{6} %ALTERAR NA DISSERTAÇÂO
	\setcounter{section}{0}
%	\input{chapter/conclusao.tex}
	
	%\newpage\null\thispagestyle{empty}\newpage
	
	%Bibliografia
	\renewcommand\bibname{Bibliografia}
	\bibliographystyle{apacite}
	\bibliography{chapter/bibliografia.tex}
	
	%Anexos
	\addcontentsline{toc}{chapter}{Anexo I - Diagrama de Gantt}
	%\includepdf[pages=-,angle=270]{chapter/pdf/Plano_de_Atividades.pdf}
%	\includepdf[pages=1,scale=0.85,angle=270,pagecommand=\chapter*{Anexo I - Diagrama de Gantt}]{chapter/pdf/Plano_de_Atividades.pdf}
\end{document}
